% Options for packages loaded elsewhere
\PassOptionsToPackage{unicode}{hyperref}
\PassOptionsToPackage{hyphens}{url}
%
\documentclass[
]{book}
\usepackage{amsmath,amssymb}
\usepackage{iftex}
\ifPDFTeX
  \usepackage[T1]{fontenc}
  \usepackage[utf8]{inputenc}
  \usepackage{textcomp} % provide euro and other symbols
\else % if luatex or xetex
  \usepackage{unicode-math} % this also loads fontspec
  \defaultfontfeatures{Scale=MatchLowercase}
  \defaultfontfeatures[\rmfamily]{Ligatures=TeX,Scale=1}
\fi
\usepackage{lmodern}
\ifPDFTeX\else
  % xetex/luatex font selection
\fi
% Use upquote if available, for straight quotes in verbatim environments
\IfFileExists{upquote.sty}{\usepackage{upquote}}{}
\IfFileExists{microtype.sty}{% use microtype if available
  \usepackage[]{microtype}
  \UseMicrotypeSet[protrusion]{basicmath} % disable protrusion for tt fonts
}{}
\makeatletter
\@ifundefined{KOMAClassName}{% if non-KOMA class
  \IfFileExists{parskip.sty}{%
    \usepackage{parskip}
  }{% else
    \setlength{\parindent}{0pt}
    \setlength{\parskip}{6pt plus 2pt minus 1pt}}
}{% if KOMA class
  \KOMAoptions{parskip=half}}
\makeatother
\usepackage{xcolor}
\usepackage{longtable,booktabs,array}
\usepackage{calc} % for calculating minipage widths
% Correct order of tables after \paragraph or \subparagraph
\usepackage{etoolbox}
\makeatletter
\patchcmd\longtable{\par}{\if@noskipsec\mbox{}\fi\par}{}{}
\makeatother
% Allow footnotes in longtable head/foot
\IfFileExists{footnotehyper.sty}{\usepackage{footnotehyper}}{\usepackage{footnote}}
\makesavenoteenv{longtable}
\usepackage{graphicx}
\makeatletter
\def\maxwidth{\ifdim\Gin@nat@width>\linewidth\linewidth\else\Gin@nat@width\fi}
\def\maxheight{\ifdim\Gin@nat@height>\textheight\textheight\else\Gin@nat@height\fi}
\makeatother
% Scale images if necessary, so that they will not overflow the page
% margins by default, and it is still possible to overwrite the defaults
% using explicit options in \includegraphics[width, height, ...]{}
\setkeys{Gin}{width=\maxwidth,height=\maxheight,keepaspectratio}
% Set default figure placement to htbp
\makeatletter
\def\fps@figure{htbp}
\makeatother
\setlength{\emergencystretch}{3em} % prevent overfull lines
\providecommand{\tightlist}{%
  \setlength{\itemsep}{0pt}\setlength{\parskip}{0pt}}
\setcounter{secnumdepth}{5}
\usepackage{booktabs}
\ifLuaTeX
  \usepackage{selnolig}  % disable illegal ligatures
\fi
\usepackage[]{natbib}
\bibliographystyle{plainnat}
\IfFileExists{bookmark.sty}{\usepackage{bookmark}}{\usepackage{hyperref}}
\IfFileExists{xurl.sty}{\usepackage{xurl}}{} % add URL line breaks if available
\urlstyle{same}
\hypersetup{
  pdftitle={청소년 금융교육},
  pdfauthor={KS Lee},
  hidelinks,
  pdfcreator={LaTeX via pandoc}}

\title{청소년 금융교육}
\author{KS Lee}
\date{2023-09-03}

\begin{document}
\maketitle

{
\setcounter{tocdepth}{1}
\tableofcontents
}
\hypertarget{about}{%
\chapter{About}\label{about}}

\textbf{김포진로체험센터}와 \textbf{고양시 청소년진로센터 }\footnote{직업인멘토링}에서 중고등 학생을 대상으로 5년간 강의한 노트를 정리 했습니다.

\hypertarget{uxc790uxb8ccuxc815uxb9ac}{%
\section{자료정리}\label{uxc790uxb8ccuxc815uxb9ac}}

신한은행에서 공개하는 PPT자료를 기본으로 쉬운내용 위주로 수업을 하였고 그러다 보니 청소년 교육을 위한 여려 교육 자료들을 수집하게 되었습니다. 미국의 FRB 웹사이트에서 제공하는 교육자료를 주의 깊게 살펴 보았고 한국은행, 청소년금융재단 등 국내의 여러 관련사이트를 통하여 교육자료를 만들 었습니다.

\hypertarget{uxd559uxc0dduxb4e4uxc758-uxc9c8uxbb38}{%
\section{학생들의 질문}\label{uxd559uxc0dduxb4e4uxc758-uxc9c8uxbb38}}

수업 중 학생들은 관심이 없는 듯 무심히 있다가도 날카로운 질문 들을 때가 가끔 하였으며 고등학교 2학년 학생 들은 자신의 진로에 대하여 고민 하는 모습을 볼 수 있었습니다:

\begin{verbatim}
1. 한국은행의 기준금리 인상이 어떻게 시중의 금리를 연관시킬 수 있는지,
2. 금융기관 직원들의 실적은 어떻게 배분 되는 지요   
\end{verbatim}

\hypertarget{uxc740uxd589-uxbca0uxc774uxc9c1}{%
\chapter{은행 베이직}\label{uxc740uxd589-uxbca0uxc774uxc9c1}}

\hypertarget{uxc740uxd589uxc5d0-uxd1b5uxc7a5uxc744-uxb9ccuxb4dcuxb294-uxc774uxc720}{%
\section{은행에 통장을 만드는 이유}\label{uxc740uxd589uxc5d0-uxd1b5uxc7a5uxc744-uxb9ccuxb4dcuxb294-uxc774uxc720}}

\begin{itemize}
\tightlist
\item
  보통예금 통장은 어떤 건가요?
\item
  적금통장
\end{itemize}

\hypertarget{uxae08uxc735uxae30uxad00uxc774-uxd558uxb294-uxc77c}{%
\section{금융기관이 하는 일}\label{uxae08uxc735uxae30uxad00uxc774-uxd558uxb294-uxc77c}}

\begin{itemize}
\tightlist
\item
  은행

  \begin{itemize}
  \tightlist
  \item
    ATM 기를 이용할 수 있다
  \item
    체크카드(debit card) 발급
  \item
    \textbf{체크카드 분실}의 경우 곧바로 은행에 신고한다. 전화 또는 은행 APP 이용하여 신고가능
  \item
    은행은 음식이나 옷 등 재화를 생산하지 않고 자동차대출(Car Loan), 주택대출(Home Mortage), 크레딧카드 서비스를 제공한다
  \item
    안전한 곳에 돈을 맡기기 위하여 또는 자동차를 구매하고 집을 장만하고 학자금대출 등을 하기 위하여 은행을 이용한다
    \\
  \end{itemize}
\item
  협동조합(Credit Union)

  \begin{itemize}
  \tightlist
  \item
    협동조합에서 통장을 만들어 예금을 하는 경우 협동조합의 소유자 중 한명이(owner-member) 된다는 것이다
  \item
    회사의 형태와 다른 의미
  \item
    조합원에게 일반적으로 예금금리는 높고 대출금리는 은행보다 낮을 수 있다
    \\
  \end{itemize}
\item
  인터넷은행

  \begin{itemize}
  \tightlist
  \item
    인터넷은행은 건물이나 지점을 만들지 않고 영업을 한다
  \item
    고정비에 대한 경비절감으로 위에서 언급한 금융기관보다 높은 예금이자, 낮은대출금리, 낮은 수수료를 부과 한다\\
  \item
    단점은 금융 서비스를 기대하며 대면으로 상담 할 장소가 없으며 대여금고 를 이용할 수 도 없다
  \item
    여러분 들은 어떤 종류의 금융기관이 던지 선택할 수 있으며 단지 그 기관이 인가를 받은 정상적 회사인지 예금자보호제도는 갖춰쳐 있는지 확인이 필요 함
  \end{itemize}
\end{itemize}

\hypertarget{uxd1b5uxc7a5uxc885uxb958}{%
\section{통장종류}\label{uxd1b5uxc7a5uxc885uxb958}}

\begin{itemize}
\tightlist
\item
  저축이 목적인 경우
\item
  대출이 필요한 경우
\item
  가계자금 또는 기업운영용도의 자금이 필요한 경우 통장을 달리하며 이용한다
\item
  Savings Account(이자가 붙는 예금) \& Checking Account (수표발행 또는 debit card 이용하는 예금)
\item
  거래를 시작 할때 통장의 종류, 예금금리, 수수료 및 서비스종류를 잘 비교해 보고 금융기관을 선택하세요
\end{itemize}

\hypertarget{uxd1b5uxc7a5uxac1cuxc124-uxc808uxcc28}{%
\section{통장개설 절차}\label{uxd1b5uxc7a5uxac1cuxc124-uxc808uxcc28}}

\begin{itemize}
\tightlist
\item
  신분증(주민등록증, 학생증, 여권 등)
\item
  이메일주소
\item
  때때로 이름이 있는 공과금 청구내역 등
\end{itemize}

\hypertarget{uxc62cuxbc14uxb978-uxc18cuxbe44uxc2b5uxad00-uxb9ccuxb4e4uxae30}{%
\section{올바른 소비습관 만들기}\label{uxc62cuxbc14uxb978-uxc18cuxbe44uxc2b5uxad00-uxb9ccuxb4e4uxae30}}

\begin{itemize}
\tightlist
\item
  ATM 에서 출금거래 내역이나 체크카드에서 돈이 인출될 때 통지내역에 주의 해야 함\\
\item
  온라인으로도 반드시 확인 한다\\
\item
  예금금액과 인출내용에 대해서 주의를 기울인다\\
\item
  \textbf{용돈이 어디에 주로 쓰였는지, 기록 하고 확인한다}
  \\
  출처 \href{https://research.stlouisfed.org/publications/page1-econ/2020/10/01/banking-basics}{Page One Economics(미국 FRB)}\\
  기타 \href{http://www.bok.or.kr/portal/main/contents.do?menuNo=201040}{청소년금융교육(한국은행)}
\end{itemize}

\hypertarget{cross}{%
\chapter{비트코인논쟁}\label{cross}}

\hypertarget{bitcoin-uxd654uxd3d0uxc778uxac00-uxd22cuxc790uxc790uxc0b0uxc778uxac00}{%
\section{Bitcoin 화폐인가 투자자산인가}\label{bitcoin-uxd654uxd3d0uxc778uxac00-uxd22cuxc790uxc790uxc0b0uxc778uxac00}}

-Money or Financial Investment

\hypertarget{bitcoin-is-money}{%
\section{Bitcoin is money}\label{bitcoin-is-money}}

\begin{itemize}
\tightlist
\item
  내용1
\end{itemize}

\hypertarget{uxd22cuxc790uxc218uxb2e8uxc774uxb2e4}{%
\section{투자수단이다}\label{uxd22cuxc790uxc218uxb2e8uxc774uxb2e4}}

\begin{itemize}
\tightlist
\item
  내용2
\end{itemize}

\hypertarget{uxacb0uxb860}{%
\section{결론}\label{uxacb0uxb860}}

\begin{itemize}
\tightlist
\item
  화폐로서 또 결제수단으로서 일부 기능을 가지고 있음
\item
  다른사람이나 업체에 송금할 수있고 심지어 해외거래도 가능
\item
  그러나 안전성(security problem)과 과도한 가격 변동성은 화폐로서 바람직하지 않음
\item
  수요와 공급에의해서 화폐가격이 형성되는 것에 비하여 비트코인은 참가자들이 미래 가격을 투기적으로 예측하여 거래 하고,
\item
  비트코인 공급은 매우 제한적인 이유로 비트코인에 대한 수요또한 변동하고 결국 가격이 요동치게 된다
\item
  이러한 과도한 가격변동 요인은 가치저장 측면에서 결점이라고 말할 수 있다
\item
  각국의 정부는 중앙은행에 화폐가지 안정정책을 위임하고 있다
\item
  화폐수요가 급변하는 경우 중앙정부는 공급을 충분히 진행하여 가치변동에 대한 제어를 하게 된다
\item
  투자자산으로서 비트코인에 대한 관심이 지대해 지며 또한 그로 인한 과도한 손실도 일어나고 있음
\item
  2018년 현재 비트코인 구매자들은 재화를 구입하는 화폐의 기능을 기대하기 보다 투자자산으로서 매입을 하고 있다
\item
  재넛 엘런재무장관은 ``비트코인은 가치저장수단도 가지고 있지 않고 법정통화로서의 구성요건도 갖추고 있지 않은 단지 매우 위험한 투기적 자산일 뿐이다'' 라고 언급 함
\item
  former Federal Reserve Chair Janet Yellen said that Bitcoin is ``not a stable source of store of value, and it doesn't constitute legal tender''; in her judgement, Bitcoin ``is a highly speculative asset.
\end{itemize}

\hypertarget{uxccaduxc18cuxb144-uxad50uxc721}{%
\chapter{청소년 교육}\label{uxccaduxc18cuxb144-uxad50uxc721}}

You can add parts to organize one or more book chapters together. Parts can be inserted at the top of an .Rmd file, before the first-level chapter heading in that same file.

Add a numbered part: \texttt{\#\ (PART)\ Act\ one\ \{-\}} (followed by \texttt{\#\ A\ chapter})

Add an unnumbered part: \texttt{\#\ (PART\textbackslash{}*)\ Act\ one\ \{-\}} (followed by \texttt{\#\ A\ chapter})

Add an appendix as a special kind of un-numbered part: \texttt{\#\ (APPENDIX)\ Other\ stuff\ \{-\}} (followed by \texttt{\#\ A\ chapter}). Chapters in an appendix are prepended with letters instead of numbers.

\hypertarget{uxd654uxd3d0uxc640-uxd1b5uxd654uxb7c9}{%
\chapter{화폐와 통화량}\label{uxd654uxd3d0uxc640-uxd1b5uxd654uxb7c9}}

\hypertarget{uxd654uxd3d0}{%
\section{화폐}\label{uxd654uxd3d0}}

\begin{itemize}
\tightlist
\item
  화폐의 기능
  시중에 돈이 지나치게 많으면 가치가 떨어져서 물가가 오르는 인플레이션이 발생 한다, 지나치게 적은 경우에는 경제활동이 위축되어 경기 침체가 발생 한다
\end{itemize}

수요와 공급 원칙에 의해서 돈이 풍부한 경우 금리가 하락하고 그 반대의 경우 금리가 오른 상태가 된다

\begin{itemize}
\tightlist
\item
  화폐로 통용되는 것 들
  물물교환 시절에는 조개가 화페로 쓰이기도 했으며 1960년대까지 우리나라는 쌀이 화폐를 대신 하기도 하였다
  조개껍질 구슬 무멸, 노예
  노예는 20세기 초까지 아프리카서부 지역에서 화폐의 기능을 가지고 있었다
\item
  공급량이 한정되어 있어서 가치의 안정성을 가지고 있는 금을 금속화폐 라고도 한다
\item
  이후 정부가 법으로 화폐의 가치를 강제적으로 부여(Fiat Money)
\end{itemize}

\hypertarget{back-to-r-uxad7f}{%
\section{back to R 굿}\label{back-to-r-uxad7f}}

\begin{itemize}
\tightlist
\item
  우리는 앞에서 중앙은행이 법화제도의 운용을 책임지고 있다는 사실을 학습하였습니다. 또한 중앙은행은 물가안정을 통해 국가의 경제가 안정적으로 성장할 수 있도록 이바지할 의무도 있습니다.\\
\item
  이와 같은 측면에서 중앙은행은 통화발행 권한을 이용하여 앞서 학습한 경기변동에 대응하여 물가안정을 도모함으로써 경기안정화를 달성하기 위해 노력합니다.
\end{itemize}

이제 중앙은행이 화폐의 공급량인 통화량을 조절함으로써 어떻게 물가를 안정시키고 경기변동에 대응하는지를 알아보겠습니다

\begin{itemize}
\tightlist
\item
  경제규모에 비해 시중에 돈이 지나치게 많으면 그 가치가 떨어져 물가가 지속적으로 오르는 인플레이션이 발생하고, 반대로 지나치게 적은 경우에는 생산자금이 부족하게 되어 경제활동이 위축되는 경기침체가 발생합니다.\\
\item
  따라서 우리나라의 경우 돈의 양을 조절하는 책임을 지고 있는 한국은행은 몇 가지 수단을 사용해서 통화량을 적절하게 조정함으로써 화폐가치의 안정, 즉 물가안정에 힘쓰고 있습니다.
\end{itemize}

\hypertarget{uxd1b5uxd654uxb7c9uxc870uxc808}{%
\section{통화량조절}\label{uxd1b5uxd654uxb7c9uxc870uxc808}}

\begin{itemize}
\tightlist
\item
  통화량을 조절하는데 있어 중요한 수단으로 활용되는 것에는 공개시장조작정책과 지급준비정책 그리고 대출정책(재할인정책) 등 3가지가 있습니다.\\
\item
  공개시장조작정책은 채권시장과 같은 금융시장에서 중앙은행이 금융기관을 상대로 국공채나 통화안정증권 등의 매매를 통해 통화량을 조절하는 정책을 말합니다.\\
\item
  중앙은행이 돈을 지불하고 민간금융기관으로부터 국공채를 매입하면 시중의 통화량이 증가하게 됩니다. 그리고 반대로 중앙은행이 돈을 받고 민간금융기관에 국공채를 매각하면 시중의 통화량이 감소하게 됩니다.\\
\item
  중앙은행이 법정지급준비율을 높이거나 낮춤으로써 통화량의 규모를 적절한 수준으로 조절하는 정책을 지급준비정책이라고 합니다.\\
\item
  그 구체적인 운용내용은 다음과 같습니다. 우리는 앞에서 은행들이 지급준비금을 보유한다는 사실을 알았고 또한 예금액 중 지급준비금으로 보유하는 비율과 예금통화 창출 규모간에는 반비례 관계가 있다는 사실도 학습했습니다.\\
\item
  그런데 중앙은행은 시중은행의 예금액 중 지급준비금 보유비율을 법령으로 규정할 수 있으며 이렇게 법령으로 강제된 지급준비금 보유비율을 법정지급준비율 이라고 합니다.\\
\item
  만일 중앙은행이 법정지급준비율을 높인다면 예금통화 창출 규모가 감소하여 시중의 통화량은 감소하게 됩니다. 반대로 중앙은행이 법정지급준비율을 낮춘다면 예금통화 창출규모가 증가하여 시중의 통화량은 증가하게 됩니다.
\end{itemize}

\hypertarget{uxd1b5uxd654uxb7c9}{%
\section{통화량}\label{uxd1b5uxd654uxb7c9}}

\begin{itemize}
\tightlist
\item
  대출정책이란 중앙은행이 민간은행에 빌려주는 자금에 대한 금리를 조정하여 통화량을 조절하는 정책을 말합니다.\\
\item
  그 구체적인 운용내용은 다음과 같습니다. 개인들이 부족한 자금을 은행에서 빌리듯이 민간 시중은행들도 부족한 자금을 중앙은행에서 빌립니다\\
\item
  그리고 중앙은행은 돈을 빌려가는 민간은행에게 부과되는 대출금리를 조정하여 통화량을 조절합니다. 이 때 대출금리는 재할인금리라고도 부릅니다\\
\item
  중앙은행이 대출금리를 낮추면 민간은행은 저렴한 비용으로 자금을 이용할 수 있게 됩니다. 따라서 중앙은행에서 더 많은 자금을 빌려갈 것이므로 시중에 통화량이 증가하게 됩니다.\\
\item
  반면에 중앙은행이 대출금리를 인상하면 민간 은행은 비싼 비용으로 자금을 이용해야 합니다\\
\item
  그러면 민간은행은 중앙은행에게 자금을 빌려가지 않거나 기존에 빌렸던 자금을 상환할 것이므로 시중에 통화량이 감소하게 됩니다
\end{itemize}

\hypertarget{uxc57duxc18duxc5b4uxc74c}{%
\section{약속어음}\label{uxc57duxc18duxc5b4uxc74c}}

\begin{itemize}
\tightlist
\item
  강원도중도개발공사(GJC)가 증권사로부터 2,030억원의 자금을 차입함
\item
  중간에는 SPC인 GJC의 자회사가 자금을 차입하고 이를 근거로 ABCP를 발행 하였음
\item
  만기일인 2022.9.8 자에 동금액을 연장(roll-over)하려 증권사를 방문 하였으나,
\item
  ABCP발행 자에 대한 보증을 섰던 지방정부인 강원도가 보증 연장에 대하여 불가 조치를 내림
\end{itemize}

\hypertarget{uxc8fcuxc2dduxacfc-uxcc44uxad8c}{%
\chapter{주식과 채권}\label{uxc8fcuxc2dduxacfc-uxcc44uxad8c}}

\hypertarget{primary-market}{%
\section{Primary market}\label{primary-market}}

발행시장이라고 하며 주식이나 채권을 회사로 부터 직접 취득 할 수 있다

\hypertarget{secondary-market}{%
\section{Secondary market}\label{secondary-market}}

프라이머리 마켓에서 취득한 투자자자산은 2차시장 즉 세컨더리 마켓에서 이를 매각할 수 있고 발행시장에 참여 하지 못했던 사람들은 여기서 이차로 투자자산을 매수 한다

주식
회사가 사업을 하려면 돈이 필요하다. 친구에게 빌릴수도 있고 지인들을 통하여 돈을 모을 수도 있다

간편한 방법은 어떤 증서를 발행하고 이를 제3자(투자자)에게 매각한다면 짧은 시간에 필요한 자금을 만들 수 있을 것이다

사업초기, 즉 회사가 초기에 자본금을 납입하고 회사설립시기나 프로젝트를 발굴하고 이를 추진 하려 할때 주식을 공모한다. 공모란 불특정 다수인에게 주식을 매각 하는 방법이다

매각된 주식은 코스닥이나 유가증권시장에 등록하고 최초 투자자는 여기서 주식을 매각하여 시세 차익을 얻을 수 있다

회사는 공모를 통하여 자금을 모았고 투자자는 주식매각을 통하여 자기의 주머니를 두둑히 하게 되었다

initial public offering (IPO). As the name implies, an IPO is a company's first sale of stock to the public.

\hypertarget{uxcc44uxad8c}{%
\section{채권}\label{uxcc44uxad8c}}

회사는 증서를 발행하고 자금을 차입한다. 채권은 타인에 대한 부채증명서인 셈이다. IOU

주식과 달리 차입금에 대하여 이자를 지급하고 언제 돈을 갚겠다고 하는 만기일이 표시 되어 있다

정부에서 도서관을 짓고 항만을 건설하고, 고속도로를 새로 계획할때 채권을 발행하여 필요한 자금을 모을 수 있다

최초로 취득한 투자자는 증권사를 통하여 이를 매각하여 시세 차익을 얻을 수 있다

\hypertarget{uxc99duxc11c-uxc885uxc774-uxc5b4uxc74c-note-paper-commercial-paper}{%
\section{증서, 종이, 어음 (note, paper, commercial paper)}\label{uxc99duxc11c-uxc885uxc774-uxc5b4uxc74c-note-paper-commercial-paper}}

금액을 표시하고, 언제 까지 지불 하겠다는 만기일을 적고 누가 이 증서를 만들었는지 회사이름을 펴기 한다음에 그 옆에 회사의 도장을 찍 는다. 이 회사가 증서의 발행인이 된다. 발행인의 신용도가 약하여 만기일에 약속한 자금을 잔환하지 못할 위험이 있는 경우 증서 매입자 들은 불 안해 할 수 있다. 이를 보완하기 위하여 보증인인 제3자 회사가 가입 할 수 있으며 이를 보증인이라고 하며 발행인과 함께 즈서에 도장을 찍고 연대하여 돈을 갚을 채무를 진다

\hypertarget{uxc0acuxc5c5uxc744-uxc704uxd55c-uxd544uxc694-uxc790uxae08}{%
\section{사업을 위한 필요 자금}\label{uxc0acuxc5c5uxc744-uxc704uxd55c-uxd544uxc694-uxc790uxae08}}

회사가 돈이 필요한 경우 몇가지 방법을 생각 해 볼 수 있다

가지고 있는 여유자금을 이용 한다. 물건을 팔아 영업을 하며 생긴 이익금을 말 한다
친구, 집안친척 또는 가까운 지인에게 부탁 한다
회사소유권을 일부 넘겨주고 투자자를 유치 한다
은행에 가서 돈을 차입한다
모르는 사람들에게 주식을 발행하여 단시간에 대량 매각 하여 자금을 모은다 이때 투자자는 일정부분의 회사지분을 취득 하게 된다
채권을 발행 한다. 만기 때 되 갚는 다

\hypertarget{uxae08uxb9acuxc778uxc0c1uxacfc-uxc804uxc138uxc0acuxae30}{%
\chapter{금리인상과 전세사기}\label{uxae08uxb9acuxc778uxc0c1uxacfc-uxc804uxc138uxc0acuxae30}}

\hypertarget{uxc804uxc138}{%
\section{전세}\label{uxc804uxc138}}

\hypertarget{uxc6d4uxc138}{%
\section{월세}\label{uxc6d4uxc138}}

\hypertarget{uxb9e4uxb9e4uxac00uxc804uxc138uxac00-uxc2e0uxcd95uxbe4cuxb77cuxc758-uxacbduxc6b0-uxb9e4uxb9e4uxac00-uxd655uxc778-uxc5b4uxb824uxc6c0}{%
\section{매매가\textless 전세가 (신축빌라의 경우 매매가 확인 어려움)}\label{uxb9e4uxb9e4uxac00uxc804uxc138uxac00-uxc2e0uxcd95uxbe4cuxb77cuxc758-uxacbduxc6b0-uxb9e4uxb9e4uxac00-uxd655uxc778-uxc5b4uxb824uxc6c0}}

학생들에게 전세사기에 대한 애기는 잠깐하고 직업교육에 대한 실무가 필요하다

\hypertarget{uxac2duxd22cuxc790}{%
\section{갭투자}\label{uxac2duxd22cuxc790}}

\hypertarget{uxc784uxb300uxc778-vs-uxc784uxcc28uxc778}{%
\section{임대인 vs 임차인}\label{uxc784uxb300uxc778-vs-uxc784uxcc28uxc778}}

\hypertarget{uxbe4cuxb77cuxc655-uxae40uxc528uxbe4cuxb77cuxc640-uxc624uxd53cuxc2a4uxd154-1100uxcc44uxc18cuxc720-uxc870uxc120uxc77cuxbcf4-22.12.27uxae30uxc0ac}{%
\section{빌라왕 김씨(빌라와 오피스텔 1,100채소유, 조선일보 22.12.27기사)}\label{uxbe4cuxb77cuxc655-uxae40uxc528uxbe4cuxb77cuxc640-uxc624uxd53cuxc2a4uxd154-1100uxcc44uxc18cuxc720-uxc870uxc120uxc77cuxbcf4-22.12.27uxae30uxc0ac}}

\hypertarget{uxc804uxc138uxae08uxd68cuxc218uxbc29uxbc95uxbcf4uxc99duxbcf4uxd5d8-uxc9d1uxc8fcuxc778uxc0acuxb9dduxc2dc-uxacbduxb9e4uxb97c-uxd1b5uxd574-uxd68cuxc218}{%
\section{전세금회수방법(보증보험-\textgreater 집주인사망시-\textgreater 경매를 통해 회수)}\label{uxc804uxc138uxae08uxd68cuxc218uxbc29uxbc95uxbcf4uxc99duxbcf4uxd5d8-uxc9d1uxc8fcuxc778uxc0acuxb9dduxc2dc-uxacbduxb9e4uxb97c-uxd1b5uxd574-uxd68cuxc218}}

이 기 성
미국 일리노이주립대학교 재무전공(UIUC)
국민대학교 경영학 박사(전공 Finance)
종합금융회사 10년, 시중은행18년, 미소금융재단 근무

수업이 끝나고 나면 ..

\hypertarget{uxc774uxc790uxc728uxc758-uxc758uxbbf8uxb97c-uxc774uxd574uxd558uxace0-uxc608uxae08-uxc801uxae08-uxd380uxb4dcuxb97c-uxad6cuxbd84-uxd560-uxc218-uxc788uxb2e4}{%
\chapter{이자율의 의미를 이해하고 예금, 적금, 펀드를 구분 할 수 있다}\label{uxc774uxc790uxc728uxc758-uxc758uxbbf8uxb97c-uxc774uxd574uxd558uxace0-uxc608uxae08-uxc801uxae08-uxd380uxb4dcuxb97c-uxad6cuxbd84-uxd560-uxc218-uxc788uxb2e4}}

수업이 끝나고 나면 금융기관의 종류를 공부하여 우리 주위에 있는 금융기관을 구별 할 수 있다. 또한 영상을 통하여 회사생활을 간접경험 한다. 금융관련 직업에 대하여 구체적인 생각을 할 수 안목이 생깁니다. 미국 월스트릿에서 요즘 핫 한 이야기를 둘러 보고 미니 경제학시간에는 인플레이션 원인가 대책 자동화시대에 바람직한 직업선택, 준비방법등에 대한 이해가 향상 됩니다

\hypertarget{uxc9c0uxae08-uxc6d4uxc2a4uxd2b8uxb9bfuxc5d0uxc11cuxb294}{%
\section{지금 월스트릿에서는\ldots{}}\label{uxc9c0uxae08-uxc6d4uxc2a4uxd2b8uxb9bfuxc5d0uxc11cuxb294}}

일론 머스크의 트위터 인수가 일어 났습니다\\
일론 머스크의 재산 2,580억불(약 300조원)

70억불(8.2조원) 파이낸싱 계획 발표, 5월 5일

\begin{itemize}
\tightlist
\item
  오라클창업자 래리엘리슨 10억불\\
\item
  가상화폐거래소 바이낸스 5억불\\
\item
  벤쳐캐피탈 세퀴아 8억불\\
\item
  모건스탠리은행 62억달러\\
\item
  카타르주권펀드\\
\item
  사우디 알왈디 왕자\\
\item
  총 18개, 은행포함 투자자 참여
\end{itemize}

\hypertarget{uxae08uxb9acuxc5d0-uxb300uxd558uxc5ec}{%
\section{금리에 대하여}\label{uxae08uxb9acuxc5d0-uxb300uxd558uxc5ec}}

금리는 무엇인가요, 이자율과 금리는 같은 내용 일까요?

\begin{itemize}
\tightlist
\item
  백만원을 은행에 예금 했을 때 1년후 1,050,000원을 받을 수 있다\\
\item
  매달 83,000원을 1년동안 적금을 하는 경우 투자한금액은 백만원, 이자는?\\
\item
  금리가 오르면 예금하는 사람 입장에서는 수익이고 돈을 빌리는 입장에서는 이자비용이 늘어납니다
\end{itemize}

중앙은행(한국)은행은 이자율을 조정하여 통화정책을 수행합니다. 경기가 좋을 때 즉 지금 처럼 TV에서 인플레를 염려 할 때는 금리를 올립니다

\begin{itemize}
\tightlist
\item
  5월26일 오전 9시 기준금리인상 뉴스, 1.5\%-\textgreater1.75\%
\end{itemize}

\hypertarget{uxae08uxc735uxae30uxad00uxc758-uxc885uxb958}{%
\section{금융기관의 종류}\label{uxae08uxc735uxae30uxad00uxc758-uxc885uxb958}}

제1금융권이 있습니다(은행)

\begin{itemize}
\tightlist
\item
  한국은행법에 의해서 운영중인 한국은행\\
\item
  특수은행법 설립된 산업은행, 한국수출입은행, 기업은행, 농협은행 등\\
\item
  일반 시중은행은 신한은행, KB, 하나은행, 지방은행인 대구 은행\\
\item
  외국은행 국내지점
\end{itemize}

그 외 금융기관(비은행, 제2금융권)

\begin{itemize}
\tightlist
\item
  상호저축은행, 새마을금고, 신용협동조합, 또 보험회사, 증권회사 리스회사 등도 바로 비은행금융기관, 즉 제2금융권에 속한다
\end{itemize}

그러면 우리가 주변에서 접할 수 있는 금융기관에는 어떤 것들이 있는지 알아보도록 하겠습니다. 금융기관은 취급하는 금융서비스의 성격에 따라 크게 은행과 비은행금융기관으로 구분됩니다.

\begin{itemize}
\tightlist
\item
  은행은 예금과 대출업무를 주로 취급하는 금융기관인데, 은행법에 의해 설립된 일반은행과 각각의 개별법에 의한 특수은행으로 대별됩니다.\\
\item
  비은행금융기관은 은행을 제외한 금융기관을 통칭하여 부르는 말로서, 은행법에 의한 적용을 받지 않으면서도 일반은행과 유사한 금융상품을 취급할 뿐만 아니라 일반은행이 취급하지 않는 다양한 금융상품을 다루는 금융기관을 말합니다.
\end{itemize}

그리고 비은행금융기관에는 보험회사, 증권회사, 종합금융회사, 자산운용회사, 상호저축은행, 신용협동기구 등이 있는데 여기서는 이들 가운데서 상대적으로 여러분에게 익숙한 금융상품을 취급하는 보험회사와 증권회사에 대해 알아보겠습니다.

\begin{itemize}
\tightlist
\item
  보험회사는 다수의 보험계약자를 상대로 보험료를 받아 이 자금을 대출, 유가증권, 부동산 등에 운용해서 보험계약자의 질병, 노후, 사망, 사고 시 보험금을 지급하는 업무를 합니다. 보험회사에는 생명보험회사, 손해보험회사 등이 있습니다.\\
\item
  증권회사는 자본시장에서 채권과 주식 등 유가증권의 발행을 주선하고 발행된 유가증권의 매매를 중개하는 업무 등을 주로 수행합니다
\end{itemize}

\hypertarget{uxae08uxc735uxc0acuxbb34uxc6d0uxc740uxd589uxc6d0uxc774-uxd558uxb294-uxc77c}{%
\section{금융사무원(은행원)이 하는 일}\label{uxae08uxc735uxc0acuxbb34uxc6d0uxc740uxd589uxc6d0uxc774-uxd558uxb294-uxc77c}}

은행원들은 한 곳에서만 근무하지 않고,
2-3년마다 이동해요!

영업점 : 고객과 만나며 상담하고, 상품을 판매해요!
본부부서 : 영업점이 잘 운영되도록 돕는 일을 해요!

\begin{itemize}
\item
  영업점
\item
  본부부서
\item
  주주명부
\end{itemize}

\hypertarget{uxbd80-uxb85d-addendum}{%
\section{부 록 addendum}\label{uxbd80-uxb85d-addendum}}

\begin{itemize}
\tightlist
\item
  직업으로서의 장점\\
\item
  금융기관이 필요로 하는 인재상\\
\item
  준비과정
\end{itemize}

\hypertarget{glossary}{%
\section{Glossary}\label{glossary}}

\begin{itemize}
\tightlist
\item
  펀드\\
\item
  최저임금\\
\item
  채권자(債權者) vs 채무자\\
\item
  투자가\\
\item
  주식\\
\item
  채권(債券,Bond)\\
\item
  벤처캐피탈회사
\end{itemize}

\begin{figure}
\centering
\includegraphics{C:/Users/Kslee5838/Pictures/stock_pic.png}
\caption{주권형태}
\end{figure}

\hypertarget{welcome}{%
\section{Welcome}\label{welcome}}

\includegraphics[width=8in]{C:/Users/Kslee5838/Pictures/stock_pic}

\includegraphics[width=8in]{C:/Users/Kslee5838/Pictures/stock_pic}

\hypertarget{uxb098uxb294-uxc5b4uxb290-uxcabd-uxd22cuxc790uxac00-uxc77cuxae4c}{%
\chapter{나는 어느 쪽 투자가 일까?}\label{uxb098uxb294-uxc5b4uxb290-uxcabd-uxd22cuxc790uxac00-uxc77cuxae4c}}

\hypertarget{uxc801uxadf9uxc801-uxd22cuxc790uxc790active-investor}{%
\section{적극적 투자자(Active Investor)}\label{uxc801uxadf9uxc801-uxd22cuxc790uxc790active-investor}}

\begin{itemize}
\tightlist
\item
  Beat the stock market
\item
  전략 : Buy, hold or sell
\item
  거래비용(Transaction cost)
\item
  펀드매니저, 연구, 예측하고, 매수, 보유, 매도에서 끊임없는 고민
\item
  주식가치=시장가격은 같지 않다
\end{itemize}

\hypertarget{uxc18cuxadf9uxc801-uxd22cuxc790uxc790passive-investorbreakfast}{%
\section{소극적 투자자(Passive Investor)Breakfast}\label{uxc18cuxadf9uxc801-uxd22cuxc790uxc790passive-investorbreakfast}}

\begin{itemize}
\tightlist
\item
  주가는 현재와 미래의 정보를 모두 반영하고 있음
\item
  전략 : Buy \& hold
\item
  주식가치=시장가격 으로 일치한다
\end{itemize}

\hypertarget{automation-and-the-minimum-wage}{%
\chapter{Automation and the Minimum Wage}\label{automation-and-the-minimum-wage}}

1938년 최저임금 도입, 프랭클린루즈벨트 대통령
당시 25센트, 궁핍을끝내기 위해 소수에게 적용
현재 주정부의 최저임금정책 7.25달러, 2009년 마지막 개정
명목(nominal) 최저임금은 계속 상승했지만 실질 구매력은 상승못함

\hypertarget{uxad50uxacfcuxc11cuxc801-uxc811uxadfcuxbc95}{%
\section{교과서적 접근법}\label{uxad50uxacfcuxc11cuxc801-uxc811uxadfcuxbc95}}

노동시장은 공급측면(노동자)와 수요측면(그들을 원하는 회사) 으로 구성 됨
공급사이드 주어진 임금하에서 일을 하려는 의사가 있음
수요사이드 주어진 임금하에서 노동자를 고용하려는 의사가 있음
균형, 일정한 가격에서 We 생성, 일하려는 근로자의 숫자와 회사가 고용하려하는 숫자가 일치함
정부에서 최저임금을 강제하는 경우, 균형가격 위에서 형성 최저임금은 price floor 역할을 하게 됨
높은 임금을 노리고 덜숙련된 노동시장에 진입하여 공급이 늘어난다. 수요자는 노동자에게 지급되는 임금이 부담스러워서 채용을 줄이게 된다
일단 높은 최저임금에서 채용된 사람들은 이익(beneficiaries)을 누리게 된다. 반면 그래프에서 보 듯이 일정 부분은 과다 공급으로 나타나게 된다
이들은 현재 일자리 보다 많은 공급으로 실업으로 나타나게 된다

\hypertarget{miss-nuancesuxbbf8uxbb18uxd55c-uxcc28uxc774}{%
\section{miss nuances(미묘한 차이)}\label{miss-nuancesuxbbf8uxbb18uxd55c-uxcc28uxc774}}

최저임금을 도입하더라도 lay off 로 이어지진 않았다. Davis Card \& Alan Kruger, 1990년대 뉴저지를 예로 들었음. 패스트푸드업계
누가 이익을 보았나?
1.1백만명이 혜택을 보았음 . 이는 시간 임금자의 1.5\%로 미미한 수준
45\%는 25살이하 젊은이, 20\%는 틴에이지. 가계의 주수입원은 아니지만 일을 통해서 경험을 쌓는다

\hypertarget{trade-off}{%
\section{Trade off}\label{trade-off}}

찬성론자

\begin{itemize}
\tightlist
\item
  저소득 임금자에게 식량, 월세 등 필요한 생활비를 충당하며 삶의 질을 높인다
\item
  예를 들어 2025년 \$25로 최저임금이 되는 경우 17백만명이 혜택을 누리고 빈곤층이 90만명 줄어 들게 된다
\item
  또한 근로자들의 사기를 진작하고 고용자의 제반비용(turn over \& hiring/training cost)를 줄인다
\end{itemize}

반대론자

\begin{itemize}
\tightlist
\item
  15달러 최저임금은 저숙련 일자리를 줄여서 거기에서 일하는 사람들이 타격을 받게 된다
\item
  그래서 1.4백만명이 일자리를 잃게 된다. 이는 0.9\%이다
\item
  경제학 적으로 근로자의 임금을 초과하는 이익(revenue)가 있을 때 일자리는 존재 할 수 있고 비로서 채용이 생긴다
\item
  최저임금을 도입하는 경우 수익(revenue)이 노동비용을 따라가지 못 할 수 있다
\end{itemize}

간단히 말해서 높은 임금을 주고 동수의 근로자를 유지는 불가능 하게 된다
결국 저숙련노동자들의 실업으로 이어지게 된다

\hypertarget{eitc-uxc0dduxb7b5}{%
\section{EITC (생략)}\label{eitc-uxc0dduxb7b5}}

\hypertarget{automation-and-the-minimum-wage-1}{%
\section{Automation and the Minimum wage}\label{automation-and-the-minimum-wage-1}}

자동화는 기구(apparatus), 공정, 어떤 기계로 인간의 노동을 대체할 수 있는 수단을 말한다
새로운 것이 아니며 산업화시대 이후 꾸준히 영향을 끼치고 있다
로봇과 AI는 지금시대의 자동화로 생각 할 수 있다. 공상과학 속의 드로이드도 있지만 간단한 알고리즘을 이용한 기구도 있다

예을 들어 집안의 온도를 조절하는 온도계도 로봇의 일종 일수 있다. 반복적인 일을 행하며 사람이 작동하는 것이 필요 없는 기계다
Physical Capital과 달리 로봇은 프로그램되어서 사람의 간섭없이 무한 반복이 가능 해진다

하는 일을 자동화 시켜보자, 우리 일은 여러가지 스텝으로 분리 할 수 있다. 반복적일 일 수록 자동화가 쉽다(susceptible to automation)

우리가 하는 일을 자동화시켜보자.

\begin{itemize}
\tightlist
\item
  일은 몇가지 구분 된 동작으로 이루 어져 있다
\item
  사람이 할수 도 있고 기계(physical capital)를 이용하여 완료 되기도 한다
\item
  정형화 되 있고 반복적 일은 좀더 자동화가 쉬울 수 있다
\item
  기술이 발전하면서 사람과 자동화된 기계가 하는 일은 좀더 복잡하게 상호작용하며 업무를 마감한다
  자동차 조립라인을 생각해보자
\item
  자동차산업 초기인 1900년대에는 작업장에 공구를 든 노동자들이 모든 과정을 담당 했다
\item
  그러나 시간이 지나면서 웰딩이나, 페인팅은 지금 자동화된 기계가 담당 하고있다
  물론 좀더 복잡한 기술이 필요한 곳은 아직도 사람에 의해서 마무리 되고 있기 도 하다
\end{itemize}

보완하고 대체한다
- 자본비용 : 기술이 발전하며 많은 부분에서 응용될 때 자본비용은 하락 하는 경향이 있다
- 노동비용 : 저숙련 노동자들의 임금은 당시의 노동시장환경이나, 정부의 정책, 또는 최저임금 때문에 높게 책정되는 경우가 있고 이는 다른 생산요소로 대체되는 경향이 있다

\hypertarget{uxc790uxb3d9uxd654uxb294-uxb2e4uxc74cuxc758-uxacbduxd5a5uxc774-uxc788uxb2e4}{%
\section{자동화는 다음의 경향이 있다}\label{uxc790uxb3d9uxd654uxb294-uxb2e4uxc74cuxc758-uxacbduxd5a5uxc774-uxc788uxb2e4}}

경제학적으로 보면 기계투입 단위당 비용이 노동투입 단위당 비용보다 저렴한 경우
노동자를 기계로 대체하려 한다

\begin{itemize}
\tightlist
\item
  marginal cost of producing goods with capital \textless{} marginal cost of producing goods with labor
\end{itemize}

최저임금상승은 회사가 새로운기술을 도입하여 기존의 노동자를 해고시키려는 유인을 제시한다

\begin{itemize}
\tightlist
\item
  즉 높은 임금은 노동비용을 상승시키고 전반적인 업무에 부담이 되는 경우 자동화로 변화되는 경우가 강하다
\item
  최저임금 직종은 대부분 노동집약적 이며, 비숙련 일로 구성(low-skill tasks) 되어있다
\item
  예를 들어 수퍼마켓의 셀프체크 레인이나 패스트푸드점의 디지털 키오스크 도입을 볼수 있다
\item
  상승된 최저임금은 비숙련 노동자들의 고용을 줄이고 이는 대부분 노령 노동자, 여성 그외 사회적 약자의 채용을 감소 시키게 된다
\end{itemize}

\hypertarget{uxbbf8uxb798uxb97c-uxc900uxbe44}{%
\section{미래를 준비}\label{uxbbf8uxb798uxb97c-uxc900uxbe44}}

인간의 노동력을 기계가 대신한다는 암울한 미래가 있지만 인간이 도리어 기계를 몰아 낼수 있는 서로 보완적인 관계가 있음을 알아야 한다

\begin{itemize}
\tightlist
\item
  저숙련노동자(low-skilled workers)을 대신하는 자리에 높은 수준의 인간노동력이 대체 될수 있다
\item
  엑셀은 계산원의 자리를 위협했지만 분석능력이 요구되는 회계처리가나 경영컨설턴트 들과 같은 높은 수준의 노동자들을 신규 창출 했다고 할 수 있다
\item
  저숙련 노동자를 해고한 제조업체에서는 기계의 문제점을 진단하고, 유지하는 고급 기술자를 채용 할 필요가 대두 되었고 그러기 위해서는 이들에게 높은 수준의
  임금을 제공 하여야 했다
\end{itemize}

우리 들은 기계의 변화에 대체되(substituted)지 않는 도리어 보완이 필요한 능력을 개발해야 한다

\begin{itemize}
\tightlist
\item
  고급기술이나 교육을 필요로 하는 분야에서 일을 할수있어야 한다. develop human capital 기술에 대체 되지 않고
\item
  권장되는 교육루트는 liberal art(논리적 사고를 기른다) 와 in science(수량적이고 기술적인 능력 함양을 위해) 의 복수 전공을 추천한다
\item
  여기서 경제학은 두 전공의 중간적인 위체에 있다고 할 수 있다. 인간과 조직의 역할을 설명하는 사회과학 분야이고 동시애 수학적 모델을 이용하여
  이론을 설명하고 테스트 하기 때문이다
\end{itemize}

\hypertarget{uxacb0uxb860-1}{%
\section{결론}\label{uxacb0uxb860-1}}

최저임금은 몇가지 논쟁을 일으킨다

\begin{itemize}
\tightlist
\item
  저임금 노동자에게 높은 임금을 주기도하지만 그 들의 고용을 감소 시키기도 한다
\item
  높아진 최저임금은 생산비용을 증가하고 이는 회사가 물건을 만드는 것을 회피하거나 조업 중단을 일으 키기도 한다
\item
  결국 상대적으로 비싸진 노동자들을 대체하여 자동화를 가속화 시키는 유인을 기업주에게 제공한다
\item
  그러나 자동화는 때로 고급인력을 필요로 해서 기회를 주기도 한다
\end{itemize}

  \bibliography{book.bib,packages.bib}

\end{document}
